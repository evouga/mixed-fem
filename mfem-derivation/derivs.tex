
\section{Energy Derivatives}

In the previous section we split our discrete mixed energy intro three different terms, and before we write our final system of equations for each Newton step, we need to sort out the necessary derivatives and energy approximations.

\subsection{Kinetic Energy}
Or kinetic energy term is
\begin{align}
E_M (\x) &= \frac{1}{2} ||\x - \tilde{\x}||_M^2 \\
\end{align}
and its derivative with respect to $\x$ is
\begin{equation}
\begin{split}
\left. \frac{\partial E_M }{\partial \x} \right|_{\x^k} &= M(\x^k - \tilde{\x}) \\
\end{split},
\end{equation}
and its hessian is
\begin{equation}
\begin{split}
\left. \frac{\partial^2 E_M }{\partial \x^2} \right|_{\x^k} &= M \\
\end{split}
\end{equation}

\subsection{Deformation Energy}
The deformation energy is typically quadratic or some higher order function of $\s$, so we take a quadratic approximation of the energy about $\s^k$, so that for $\s=\s^k + \ds$ we have

\begin{equation}
E_\Psi (\s) \approx \tilde{E}_\Psi (\s) = h^2 \sum_{K \in \Th} \left(\Psi(\s_K^k) +
 \ds_K^T \frac{\partial \Psi}{\partial \s_K}
 + \frac{1}{2}\ds_K^T \frac{\partial^2 \Psi}{\partial \s_K^2} \ds_K \right) V_K
\end{equation}
where the derivatives are evaluated at $\s^k$.
%$\gs_K =\left.\frac{\partial \Psi}{\partial s_K}\right|_{\s_K^k}$ and
%$\Hsk=\left.\frac{\partial^2 \Psi}{\partial s_K^2}\right|_{s_K^k}$, and its derivative is simply
%\begin{equation}
%\frac{\partial \tilde{E}_\Psi}{\partial \ds_K} =
%\sum_{K \in \Th} \left(\gs_K + \Hsk \ds_K \right) V_K
%\end{equation}

\subsection{Constraint Energy}

First we have the original constraint energy
\begin{equation}
E_\Lambda (\x, \s, \la) = h^2 \sum_{K \in \Th} \left( \la_K \cdot (W_K \s_K - J_K \x) \right) V_K,
\end{equation}
and now we need to derive its derivatives.
The main difficulty in these derivatives is dealing with the derivative of the rotation term.
We do this by defining the rotation in terms of the SVD of the deformation gradient,
\begin{equation}
R(\x) = U_K(\x) V_K(\x)^T \quad \text{where} \; F_K(\x) = U_K \Sigma_K V_K^T,
\end{equation}
and
\begin{equation}
\frac{\partial R_K}{\partial F_{ij}} = \frac{\partial U_K}{\partial F_{ij}}V_K^T
+ U_K \frac{\partial V_K^T}{\partial F_{ij}}.
\end{equation}

We now temporarily switch to indicial notation, and make $K$ a superscript. Our derivative is a tensor $\frac{\partial R_{ij}}{\partial F_{kl}} \in \R^{3 \times 3 \times 3 \times 3}$. Since our degrees of freedom are all written as vectors, we redefine this tensor by flattening the first two indices, and modifying the last two dimensions so that it has the effect of multiplying by a symmetric matrix and flattening the result. This gives us the tensor $\hat{W}^K_{ijk} \in \R^{9 \times 9 \times 6}$. So in the same way that we have
\begin{equation}
W^K_{ij}s^K_j = \text{vec}(R^K_{kl} S^K_{lm}),
\end{equation}
we also have
\begin{equation}
\hat{W}^K_{ijk}s^K_{k} = \frac{\partial \text{vec}(R^K_{lm}S^K_{mn})}{\partial F_{(kl)}}.
\end{equation}
The constraint energy in indicial notation for a single element is 
\begin{equation}
\El^K = h^2 \la_j^K(W^K_{jk}\s^K_k - J^K_{ij}\x_i),
\end{equation}
and the relevant derivatives are
\begin{align}
&\frac{\partial \El^K}{\partial \x_i} = 
J_{ij}(\hat{W}^K_{jkl}\s^K_l - I_{jk})\la^K_k \\
&\frac{\partial \El^K}{\partial \s^K_i}  = 
W_{ij}^K \la^K_{j} \\
&\frac{\partial \El^K}{\partial \la^K_i} = 
W_{ij}^K \s^{K}_j - J_{ij}^K \x_j \\
&\frac{\partial^2 \El^K}{\partial \la^K_i \partial \s^K_j} =
W^K_{ij} \\
&\frac{\partial^2 \El^K}{\partial \x_i \partial \la^K_k} =
J_{ij}(\hat{W}^K_{jkl}\s^K_l - I_{jk}) \\
&\frac{\partial^2 \El^K}{\partial \x_i \partial \s_l^K} =
J_{ij}\hat{W}^K_{jkl}\la^K_k \\
&\frac{\partial^2 \El^K}{\partial \x_i^2} =
\text{i'll get around to it} \\
\end{align}
where $I_{jk}$ is a $9 \times 9$ identity matrix. We omitted the $h^2$ factor from the above derivatives to make things cleaner too.



\subsection{Penalty Energy Derivatives}
The last energy term is the penalty energy. We start with the  equation
\begin{equation}
E_\kappa(\x,\s) = h^2 \frac{\mu}{2} (W(\x)\s - J\x)^T V (W(\x)\s - J\x),
\end{equation}
and switch to indicial notation in the same way we did for the constraint energy so that we have
\begin{equation}
E_\kappa(\x_i,\s_j) = h^2 \frac{\mu}{2} (W_{jk}\s_j - J_{ik}\x_i) V_{kl} (W_{lm}\s_m - J_{ln}\x_n).
\end{equation}
Its derivatives are (again we omit the $h^2$ factor)
\begin{align}
\frac{\partial E_\kappa}{\partial \s_j} &= \mu W_{jk} V_{kl} (W_{lm}\s_m - J_{ln}\x_n) \\
\frac{\partial E_\kappa}{\partial \x_i} &= \mu \left(J_{im}\hat{W}_{mjk}\s_j - J_{ik}\right) V_{kl} (W_{lm}\s_m - J_{ln}\x_n)  \\
\frac{\partial^2 E_\kappa}{\partial \s_j^2} &= \mu W_{jk} V_{kk} W_{jk} \\
\frac{\partial^2 E_\kappa}{\partial \x_i \partial \s_j} &= \mu \left( 
	J_{im}\hat{W}_{mjk}V_{kl} (W_{lm}\s_m - J_{ln}\x_n)	
  + \left(J_{im}\hat{W}_{mjk}\s_j - J_{ik}\right) V_{kl} W_{lj}
   \right)\\
\frac{\partial^2 E_\kappa}{\partial \x^2_i} &= \mu \left( 
	\underbrace{
		J_{im} \frac{\partial \hat{W}_{mjk} \s_j}{\partial \x_i} V_{kl} (W_{lm}\s_m - J_{ln}\x_n)	
		}_\text{Ignoring for now}
  + \left(J_{im}\hat{W}_{mjk}\s_j - J_{ik}\right) V_{kl} \left( J_{im}\hat{W}_{mjl}\s_j - J_{il}\right)
   \right).
\end{align}
Right now we currently ignore the second derivative of the rotation for $\frac{\partial^2 E_\kappa}{\partial \x^2_i}$.