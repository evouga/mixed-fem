\section{Problem Setup}

\subsection{Continuous Problem}
We begin with our continuous mixed energy. We write the energy as an integral over a reference domain $\Omega \subset \R^d$. For notation we denote vectors by bold lower case letters, matrices by upper-case letters, and scalars by lower case letters. Our mixed energy is a function of deformed coordinates, $\x \in \R^d$, symmetric deformation matrices $S \in \R^{d \times d}$, and Lagrange multipliers $\Lambda \in \R^{d \times d}$. Our continuous mixed variational is 

\begin{equation}
\mathcal{L}(\x,S,\Lambda) = \int_\Omega \frac{\rho}{2h^2} ||\x - \tilde{\x}||_2^2 
+ \Psi(S) - \Lambda : (R(\x)S - F(\x)) dV
\end{equation}
where $h$ is the timestep, $\rho$ is the density, and $\tilde{\x} = \x^t + h\dot{\x}^t + h^2 \fext$. In the mixed formulation, our strain energy density function $\Psi$ is now a function of $S$ rather than the deformed coordinates. $R$ is the rotation matrix from the polar decomposition of the deformation gradient, $F(\x) = R(\x)S(\x)$.

\subsection{Discretized Energy}

We represent the domain by $\mathcal{T} \equiv \{K_i\}_{i=1}^{n_e}$ consisting of a set of elements $K_1,\dots,K_{n_e}$.
Per-element variables are represent by a subscript $K$. Our deformed coordinates are discretized on a nodal mesh, such as triangles or tetrahedra, and are stored in the vector $\x \in \R^{dn}$ for $n$ nodes.
From this point on we'll assume we're working with linear tetrahedral elements in 3D, so we will have a single integration point per element and the integration weight just becomes the volume of the element, $V_K$. Using an implicit euler time integrator, our discrete variational form for the timestep $t+1$ is

\begin{equation}
\x^{t+1}, \s^{t+1}, \la^{t+1} = \argmin_{\x,\s,\la} \frac{1}{2h^2}||\x - \tilde{\x}||_M^2 
+ \sum_{K \in \Th} \left( \Psi(\s_K) - \la_K \cdot (W_K \s_K - J_K \x) \right) V_K.
\end{equation}
There's several changes here compared to the the continuous form so we explain them one-by-one. The first term, the kinetic energy term, is written now as a difference in the $M$-norm where $M$ is the mass matrix for our mesh. Next we see that $S$ is now a vector $\s \in \R^{6n_e}$ where $\s_K \in \R^6$ because there are only 6 degrees of freedom for a $3 \times 3$ symmetric matrix. Similarly our Lagrange multipliers are now $\la \in \R^{9n_e}$ where each element's Lagrange multipliers are flattened into a vector $\la_K$. Another new piece is the matrix $W_K$ which equals
\begin{equation}
W_K s_K = \text{vec}(R_K S_K),
\end{equation}
and so it is constructed from $R_K$. $W_K$ is of size $9 \times 6$ because it has the effect of performing a matrix multiplication between the rotation matrix and a symmetric matrix, then flattening the result. To simplify notation we just write $W_K$ and $R_K$ but to be more precise we would write $W_K(F_K(\x))$ and $R_K(F_K(\x))$ to indicate that the rotation is a function of the deformation gradient for the current positional configuration. Lastly we have rewritten the deformation gradient as
\begin{equation}
J_K \x = \text{vec}(F_K(\x))
\end{equation}
where $J_K \in \R^{9 \times dn}$ is a Jacobian matrix. 

Next to make our energy more suitable for an iterative Newton-like method, we note that
\begin{align*}
\x^{k+1} &= \x^k + \dx \\
\s^{k+1} &= \s^k + \ds \\
\la^{k+1} &= \la^k + \dl,
\end{align*}
where $k$ corresponds to the current Newton iterate. With this we can instead solve for updates of our variables. Our discrete mixed energy becomes
\begin{equation}
\begin{split}
E(\x^{k+1}, \s^{k+1}, \la^{k+1}) &= \frac{1}{2h^2}||\x^{k+1} - \tilde{\x}||_M^2 \\
&+ \sum_{K \in \Th} \Psi(\s_K^k + \ds_K) V_K  \\
&-  \sum_{K \in \Th} (\la_K^k + \dl_K) \cdot (W_K (\s_K^k + \ds_K) - J_K(\x^k + \dx)) V_K.
\end{split}
\end{equation}
and $W_K$ in the final term is a function of $\x^k + \dx$. 

To simplify things for derivations we split up the three terms in this energy to get
\begin{equation}
E(\x, \s, \la) = E_M (\x) + E_\Psi (\s) - \El (\x, \s, \la),
\end{equation}
and to arrive at a Newton step we just need to linearize with respect to the $\delta$ variables and construct our system of equations. The only hurdle to doing this is that $E_\Psi$ is potentially high order, and $E_\Lambda$ is nonlinear in $\x$ from the rotations, so we need to form quadratic approximations of these energies.