\documentclass[letterpaper,12pt]{article}
\usepackage[margin=0.65in]{geometry}
\usepackage{amsmath,amsthm,amsfonts,amssymb}
\usepackage{mathtools}
%\usepackage{algorithm,algpseudocode}
\usepackage{hyperref}
\usepackage{graphicx}
\usepackage{listings}
\usepackage{subcaption}
\usepackage[section]{placeins}
\usepackage{algorithm2e}


\theoremstyle{remark}
\newtheorem{claim}{Claim}

\DeclarePairedDelimiter\abs{\lvert}{\rvert}
\DeclarePairedDelimiter\floor{\lfloor}{\rfloor}
\DeclarePairedDelimiter\ceiling{\lceil}{\rceil}

\usepackage{xcolor}
 
\definecolor{codegreen}{rgb}{0,0.6,0}
\definecolor{codegray}{rgb}{0.5,0.5,0.5}
\definecolor{codepurple}{rgb}{0.58,0,0.82}
\definecolor{backcolour}{rgb}{0.95,0.95,0.92}
 
\lstdefinestyle{mystyle}{
    backgroundcolor=\color{backcolour},
    numbers=left,
    commentstyle=\color{codegreen},
    keywordstyle=\color{magenta},
    numberstyle=\tiny\color{codegray},
    stringstyle=\color{codepurple},
    basicstyle=\ttfamily\footnotesize,
    breakatwhitespace=false,         
    breaklines=true,                 
    captionpos=b,                    
    keepspaces=true,                 
    showspaces=false,                
    showstringspaces=false,
    showtabs=false,                  
    tabsize=1
}
 
\lstset{style=mystyle}


\newcommand{\V}{\mathcal{V}}
\newcommand{\HO}{ {H^1(\Omega)} }
\newcommand{\LO}{ {L^2(\Omega)} }
\newcommand{\LG}{ {L^2(\Gamma)} }
\newcommand{\LGin}{ {L^2(\Gin)} }
\newcommand{\LGout}{ {L^2(\Gout)} }
\newcommand{\R}{\mathbb{R}}
\newcommand{\Th}{\mathcal{T}}
\newcommand{\X}{\bar{\mathbf{x}}}
\newcommand{\x}{\mathbf{x}}
\newcommand{\C}{\mathbf{c}}
\newcommand{\s}{\mathbf{s}}
\newcommand{\la}{\mathbf{\lambda}}
\newcommand{\dx}{\delta \x}
\newcommand{\ds}{\delta \s}
\newcommand{\dl}{\delta \la}
\newcommand{\fext}{\mathbf{f}_\text{ext}}
\DeclareMathOperator*{\argmin}{arg\,min} 
\newcommand{\El}{E_\Lambda}
\newcommand{\gx}{ {\mathbf{g}_\x^k} }
\newcommand{\gs}{ {\mathbf{g}_\s^k} }


\RestyleAlgo{ruled}
\LinesNumbered

\begin{document}


\title{Mixed FEM stuff}
\date{}
\maketitle

\section{Problem Setup}

\subsection{Continuous Problem}
We begin with our continuous mixed energy. We write the energy as an integral over a reference domain $\Omega \subset \R^d$. For notation we denote vectors by bold lower case letters, matrices by upper-case letters, and scalars by lower case letters. Our mixed energy is a function of deformed coordinates, $\x \in \R^d$, symmetric deformation matrices $S \in \R^{d \times d}$, and Lagrange multipliers $\Lambda \in \R^{d \times d}$. Our continuous mixed variational is. \danny{(Note: this is a funny semi-discretization - we've discretized in time so far -- using Backwards Euler but not space - notice a lo of FE papers do the opposite to get an ODE first by discretizing in space first. No reason we need to start with this vs the other - right? Also we could propose the setup in the fully continuous setting first too - right?)} 

\begin{equation}
\mathcal{L}(\x,S,\Lambda) = \int_\Omega \frac{\rho}{2h^2} ||\x - \tilde{\x}||_2^2 
+ \Psi(S) - \Lambda : (R(\x)S - F(\x)) dV
\end{equation}
where $h$ is the timestep, $\rho$ is the density, and $\tilde{\x} = \x^t + h\dot{\x}^t + h^2 \fext$. In the mixed formulation, our strain energy density function $\Psi$ is now a function of $S$ rather than the deformed coordinates. $R$ is the rotation matrix from the polar decomposition of the deformation gradient, $F(\x) = R(\x)S(\x)$. \danny{Where specifically we have $R(F(x))$ to indicate $R$ is a projection of the deformation gradient onto SO(3)).}

\subsection{Discretized Energy}

We represent the domain by $\mathcal{T} \equiv \{K_i\}_{i=1}^{n_e}$ consisting of a set of elements $K_1,\dots,K_{n_e}$.
Per-element variables are represent by a subscript $K$. Our deformed coordinates are discretized on a nodal mesh, such as triangles or tetrahedra, and are stored in the vector $\x \in \R^{dn}$ for $n$ nodes.
From this point on we'll assume we're working with linear tetrahedral elements in 3D, so we will have a single integration point per element and the integration weight just becomes the volume of the element, $V_K$.\danny{($V_K$ is just a scalar - right? If so can we make it lower case?)} Using an implicit euler time integrator, our discrete variational form for the timestep $t+1$ is

\begin{equation}
\x^{t+1}, \s^{t+1}, \la^{t+1} = \argmin_{\x,\s,\la} \frac{1}{2h^2}||\x - \tilde{\x}||_M^2 
+ \sum_{K \in \Th} \left( \Psi(\s_K) - \la_K \cdot (W_K \s_K - J_K \x) \right) V_K.
\end{equation}
There's several changes here compared to the the continuous form so we explain them one-by-one. The first term, the kinetic energy term, is written now as a difference in the $M$-norm where $M$ is the mass matrix for our mesh. Next we see that $S$ is now a vector $\s \in \R^{6n_e}$ where $\s_K \in \R^6$ because there are only 6 degrees of freedom for a $3 \times 3$ symmetric matrix. Similarly our Lagrange multipliers are now $\la \in \R^{9n_e}$ where each element's Lagrange multipliers are flattened into a vector $\la_K$. Another new piece is the matrix $W_K$ which equals
\begin{equation}
W_K s_K = \text{vec}(R_K S_K),
\end{equation}
and so it is constructed from $R_K$. $W_K$ is of size $9 \times 6$ because it has the effect of performing a matrix multiplication between the rotation matrix and a symmetric matrix, then flattening the result. To simplify notation we just write $W_K$ and $R_K$ but to be more precise we would write $W_K(F_K(\x))$ and $R_K(F_K(\x))$ to indicate that the rotation is a function of the deformation gradient for the current positional configuration. \danny{(More messy looking but I'd prefer to write those dependencies out explicitly in our energies to avoid bugs and misunderstanding downstream.)} Lastly we have rewritten the deformation gradient as
\begin{equation}
J_K \x = \text{vec}(F_K(\x))
\end{equation}
where $J_K \in \R^{9 \times dn}$ is a Jacobian matrix. 

Next to make our energy more suitable for an iterative Newton-like method, we note that
\begin{align*}
\x^{t+1} &= \x^t + \dx \\
\s^{t+1} &= \s^t + \ds \\
\la^{t+1} &= \la^t + \dl,
\end{align*}
so that we can instead solve for updates of our variables. Our discrete mixed energy becomes
\begin{equation}
\begin{split}
E(\x^{t+1}, \s^{t+1}, \la^{t+1}) &= \frac{1}{2h^2}||\x^t + \dx - \tilde{\x}||_M^2 \\
&+ \sum_{K \in \Th} \Psi(\s_K^t + \ds_K) V_K  \\
&-  \sum_{K \in \Th} (\la_K^t + \dl_K) \cdot (W_K (\s_K^t + \ds_K) - J_K(\x^t + \dx)) V_K.
\end{split}
\end{equation}
and $W_K$ in the final term is a function of $\x^t + \dx$. 

To simplify things for derivations we split up the three terms in this energy to get
\begin{equation}
E(\x, \s, \la) = E_M (\x) + E_\Psi (\s) - \El (\x, \s, \la),
\end{equation}
and to arrive at a Newton step we just need to linearize with respect to the $\delta$ variables and construct our system of equations. The only hurdle to doing this is that $E_\Psi$ is potentially high order, and $E_\Lambda$ is nonlinear in $\x$ from the rotations, so we need to form quadratic approximations of these energies. \danny{Last two sentences are a bit contradictory I think? Newton process is no more/no less than just forming a quadratic approximation (not linear) via taylor expansion and then solving it. Probably good for us to confirm we're thinking of the same thing here before we get into the nitty gritty :). Also, for creating a Newton solver I'm not sure there's a need to create this bunch of extra $\delta$ terms in the above and below? Probably makes it messier then we need it to be for sanity checking.}

\section{Energy Derivatives}

In the previous section we split our discrete mixed energy intro three different terms, and before we write our final system of equations for each Newton step, we need to sort out the necessary derivatives and energy approximations.

\subsection{Kinetic Energy}
Or kinetic energy term is
\begin{align}
E_M (\x) &= \frac{1}{2h^2} ||\x - \tilde{x}||_M^2 \\
\end{align}
and its derivative with respect to $\x$ is
\begin{equation}
\begin{split}
\left. \frac{\partial E_M }{\partial \x} \right|_{\x^k} &= \frac{1}{h^2}M(\x^k - \tilde{\x}) \\
&=  \gx
\end{split},
\end{equation}
and its hessian is
\begin{equation}
\begin{split}
\left. \frac{\partial^2 E_M }{\partial \x^2} \right|_{\x^k} &= \frac{1}{h^2}M \\
&=  \Hx
\end{split}
\end{equation}

\subsection{Deformation Energy}
The deformation energy is typically quadratic or some higher order function of $\s$, so we take a quadratic approximation of the energy about $\s^k$, so that for $\s=\s^k + \ds$ we have

\begin{equation}
E_\Psi (\s) \approx \tilde{E}_\Psi (\s) = \sum_{K \in \Th} \left(\Psi(\s_K^k) + \ds_K^T \gs_K + \frac{1}{2}\ds_K^T \Hsk \ds_K \right) V_K
\end{equation}
where 
$\gs_K =\left.\frac{\partial \Psi}{\partial s_K}\right|_{\s_K^k}$ and
$\Hsk=\left.\frac{\partial^2 \Psi}{\partial s_K^2}\right|_{s_K^k}$, and its derivative is simply
\begin{equation}
\frac{\partial \tilde{E}_\Psi}{\partial \ds_K} =
\sum_{K \in \Th} \left(\gs_K + \Hsk \ds_K \right) V_K
\end{equation}


\subsection{Constraint Energy}

First we have the original constraint energy
\begin{equation}
E_\Lambda (\x, \s, \la) = \sum_{K \in \Th} \left( \la_K \cdot (W_K \s_K - J_K \x) \right) V_K,
\end{equation}
where we have $\x = \x^k + \dx$, $\s = \s^k + \ds$, and $\la = \la^k + \dl$. And for its quadratic approximation, $\tilde{E}_\Lambda$ we have

\begin{equation}
\begin{split}
\tilde{E}_\Lambda (\x, \s, \la) &=  \El (\x^k,\s^k,\la^k) 
+ \sum_{K \in \Th} \left( 
  \dx^T \frac{\partial \El^K}{\partial \x} 
+ \ds_K^T \frac{\partial \El^K}{\partial \s_K}
+ \dl_K^T \frac{\partial \El^K}{\partial \la_K} \right. \\
&+ \left. \dl_K^T \frac{\partial^2 \El^K}{\partial \la_K \partial \s_K} \ds_K
+ \dl_K^T \frac{\partial^2 \El^K}{\partial \la_K \partial \x} \dx
+ \dx^T \frac{\partial^2 \El^K}{\partial \x \partial \s_K} \ds_K
+ \frac{1}{2} \dx^T \frac{\partial^2 \El^K}{\partial \x^2} \dx
 \right) V_K
\end{split}
\end{equation}
where each of the derivatives is evaluated at $\x^k, \s^k, \la^k$. The main difficulty in these derivatives is dealing with the derivative of the rotation term. We do this by defining the rotation in terms of the SVD of the deformation gradient,
\begin{equation}
R(\x) = U_K(\x) V_K(\x)^T \quad \text{where} \; F_K(\x) = U_K \Sigma_K V_K^T,
\end{equation}
and
\begin{equation}
\frac{\partial R_K}{\partial F_{ij}} = \frac{\partial U_K}{\partial F_{ij}}V_K^T
+ U_K \frac{\partial V_K^T}{\partial F_{ij}}.
\end{equation}

We now temporarily switch to indicial notation, and make $K$ a superscript. Our derivative is a tensor $\frac{\partial R_{ij}}{\partial F_{kl}} \in \R^{3 \times 3 \times 3 \times 3}$. Since our degrees of freedom are all written as vectors, we redefine this tensor by flattening the first two indices, and modifying the last two dimensions so that it has the effect of multiplying by a symmetric matrix and flattening the result. This gives us the tensor $\hat{W}^K_{ijk} \in \R^{9 \times 9 \times 6}$. So in the same way that we have
\begin{equation}
W^K_{ij}s^K_j = \text{vec}(R^K_{kl} S^K_{lm}),
\end{equation}
we also have
\begin{equation}
\hat{W}^K_{ijk}s^K_{k} = \frac{\partial \text{vec}(R^K_{lm}S^K_{mn})}{\partial F_{(kl)}}.
\end{equation}
The constraint energy in indicial notation for a single element is 
\begin{equation}
\El^K = \la_j^K(W^K_{jk}\s^K_k - J^K_{ij}\x_i),
\end{equation}
and the relevant derivatives are
\begin{align}
&\frac{\partial \El^K}{\partial \x_i} = 
J_{ij}(\hat{W}^K_{jkl}\s^K_l - I_{jk})\la^K_k \\
&\frac{\partial \El^K}{\partial \s^K_i}  = 
W_{ij}^K \la^K_{j} \\
&\frac{\partial \El^K}{\partial \la^K_i} = 
W_{ij}^K \s^{K}_j - J_{ij}^K \x_j \\
&\frac{\partial^2 \El^K}{\partial \la^K_i \partial \s^K_j} =
W^K_{ij} \\
&\frac{\partial^2 \El^K}{\partial \x_i \partial \la^K_k} =
J_{ij}(\hat{W}^K_{jkl}\s^K_l - I_{jk}) \\
&\frac{\partial^2 \El^K}{\partial \x_i \partial \s_l^K} =
J_{ij}\hat{W}^K_{jkl}\la^K_k \\
&\frac{\partial^2 \El^K}{\partial \x_i^2} =
\text{i'll get around to it} \\
\end{align}
where $I_{jk}$ is a $9 \times 9$ identity matrix.

%
\section{System of Equations}
Now to setup our system of equations, we differentiate our energy with respect to the $\delta$-variables and set equal to zero. We switch back to vector notation and make $K$ a subscript to simplify the notation.

\subsection{Equation for $\ds$}
We start by differentiating $\tilde{E} = E_M + \tilde{E}_\Psi + \tilde{E}_\Lambda$ with respect to $\ds$:
\begin{equation}
\begin{split}
\frac{\partial \tilde{E}^K}{\partial \ds_K} &= \frac{\partial \tilde{E}^K_\Psi}{\partial \ds_K} - \frac{\partial \tilde{E}^K_\Lambda}{\partial \ds_K} \\
&= g_K^t + H_K^t \ds_K - {W_K^t}^T (\la_K^t + \dl_K) - (\hat{W}_K^{t}\cdot \la_K^t)^T J_K \dx
\end{split}
\end{equation}
where $(\hat{W}_K^{t}\cdot \la_K^t)$ performs a reduction over the middle index of $\hat{W}^K_{ijk}$. Setting this derivative equal to zero and solving for $\ds_K$ gives
\begin{equation}
\ds_K = {H_K^t}^{-1} \left( {W_K^t}^T (\la_K^t + \dl_K) + (\hat{W}_K^{t}\cdot \la_K^t)^T J_K \dx - g_K^t \right).
\end{equation}

\subsection{Equation for $\dl$}
In the same fashion we differentiate $\tilde{E}$ with respect to $\dl$:

%To simplify our system of equations we substitute the solution for $\ds$ into into $\tilde{E}$ so that we end up with a KKT system solving for $\dx$ and $\dl$. We first note that
%\begin{equation}
%\frac{\partial \ds_K}{\partial \dl_K} = W_K^t{H_K^t}^{-1},
%\end{equation}
%so now we write
%\begin{equation}
%\begin{split}
%\frac{\partial \tilde{E}^K}{\partial \dl_K} &= \frac{\partial \tilde{E}^K_\Psi}{\partial \ds_K} - \frac{\partial \tilde{E}^K_\Lambda}{\partial \ds_K} \\
%&= W_K^t{H_K^t}^{-1} \left( g_K^t + H_K^t \ds_K \right) 
%- {W_K^t}^T (\la_K^t + \dl_K) - (\hat{W}_K^{t}\cdot \la_K^t)^T J_K \dx
%\end{split}
%\end{equation}
\begin{equation}
\begin{split}
\frac{\partial \tilde{E}^K}{\partial \dl_K} &= - \frac{\partial \tilde{E}^K_\Lambda}{\partial \dl_K} \\
&= -\frac{\partial \El^K}{\partial \la_K} - \frac{\partial^2 \El^K}{\partial \s_K \partial \la_K}\ds - \frac{\partial^2 \El^K}{\partial \x \partial \la_K}\dx \\
&= (J_K\x^t - W_K^ts_K^t) - W_K^t \ds - (\hat{W}_K^{t}\cdot \s_K^t - I)^T J_K \dx \\
&= (J_K(\x^t + \dx) - W_K^t (s_K^t + \ds_K)) - (\hat{W}_K^{t}\cdot \s_K^t)^T J_K \dx \\
\end{split}
\end{equation}
where $(\hat{W}_K^{t}\cdot \s_K^t)$ performs a reduction over the last index of $\hat{W}^K_{ijk}$.

\subsection{Equation for $\dx$}
Last we differentiate $\tilde{E}$ with respect to $\dx$ (we omit $\frac{\partial^2 \El^K}{ \partial \x^2} \dx$ for now):

\begin{equation}
\begin{split}
\frac{\partial \tilde{E}}{\partial \dx} &= \frac{\partial E_M }{\partial \dx} - \sum_{K \in \Th} \left(
  \frac{\partial \tilde{E}^K_\Lambda}{\partial \x} 
+ \frac{\partial^2 \El^K}{\partial \la_K \partial \x} \dl_K
+ \frac{\partial^2 \El^K}{\partial \s_K  \partial \x} \ds_K
\right) V_K \\
&=  \frac{1}{h^2}M(\dx - h\dot{\x}^t - h^2 \fext) \\
&- \sum_{K \in \Th} \left(
J_K^T(\hat{W}_K^{t}\cdot \s_K^t - I)(\la_K^t + \dl_K) +
J_K^T(\hat{W}_K^{t}\cdot \la_K^t) \ds_K
\right)V_K
\end{split}
\end{equation}

\subsection{System of Equations}
To build the system of equations we define some variables to keep things neat:
\begin{align}
G &= J^w - \sum_{K \in \Th} {P_K^\la}^T \left((\hat{W}_K^{t}\cdot \s_K^t)^T J_K^w \right) \\
D &= -\sum_{K \in \Th} {P_K^\la}^T  W_K^t P_K^\s\\
F &= - \sum_{K \in \Th} {P_K^\s}^T \left((\hat{W}_K^{t}\cdot \la_K^t)^T J_K^w \right)
\end{align}
where $J^w$ is the volume weighted Jacobian, and $P_K^\la \in \R^{9 \times 9n_e}$ and $P_K^\s \in \R^{6 \times 6n_e}$ are selection matrics for the $K$-th set of Lagrange multipliers and deformation variables, respectively. Additionally $H$ is the block diagonal Hessian for $s^t$. We form the system by setting each of the derivatives with respect to the $\delta$-variables equal to zero and get

\begin{equation}
\begin{pmatrix}
\frac{1}{h^2}M & F^T & G^T \\
F & H & D^T\\
G & D & 0
\end{pmatrix}
\begin{pmatrix}
\dx \\
\ds \\
\dl
\end{pmatrix} =
\begin{pmatrix}
\frac{1}{h^2}M\gx - G^T \la^t \\
 -g^t - D^T \la^t \\
-J^w\x^t -D \s^t 
\end{pmatrix}.
\end{equation}

Next, to get this thing in a KKT-like form we substitute the solution for the $\ds$ into the other equations. So we have
\begin{equation}
\ds = H^{-1}\left(-g^t - D^T(\la^t + \dl) - F\dx \right),
\end{equation}
and sorry cause it's about to get a little ugly, but for the first set of equations we have
\begin{equation*}
M\dx + F^TH^{-1}\left(-\gs - D^T(\la^t + \dl) - F\dx \right) +G^T \dl = \frac{1}{h^2}M\gx - G^T\la^t
\end{equation*}
and collect terms to get
\begin{equation}
(M - F^TH^{-1}F)\dx + (G^T - F^TH^{-1}D^T)\dl = 
\frac{1}{h^2}M\gx + F^TH^{-1}\gs + (F^TH^{-1}D^T - G)\la^t.
\end{equation}
I promise it gets cleaner, but next we substitute the solution for $\ds$ into the equations for $\dl$:
\begin{equation*}
G\dx + DH^{-1}\left(-\gs - D^T(\la^t + \dl) - F\dx\right) = -J^w\x^t - D\s^t,
\end{equation*}
and collect terms so that we have
\begin{equation}
(G - DH^{-1}F)\dx - DH^{-1}D^T\dl = -J^w\x^t + D(H^{-1}\gs - s^t + H^{-1}D^T\la^t).
\end{equation}
Before moving the equations back into matrix form, we define the following variables:
\begin{align*}
\tilde{M} &= (\frac{1}{h^2}M - F^TH^{-1}F) \\
\tilde{J} &= (G - DH^{-1}F). \\
\end{align*}

Now our final system of equations for a single Newton step is
\begin{equation}
\begin{pmatrix}
\tilde{M} & \tilde{J}^T \\
\tilde{J} & -DH^{-1}D^T 
\end{pmatrix}
\begin{pmatrix}
\dx \\
\dl
\end{pmatrix} =
\begin{pmatrix}
\frac{1}{h^2}M\gx + F^TH^{-1}\gs + (F^TH^{-1}D^T - G^T)\la^t \\
-J^w\x^t + D(H^{-1}\gs - s^t + H^{-1}D^T\la^t)
\end{pmatrix}.
\end{equation}

\section{Energy Proxy}
Let $c_K(\x,\s) = W_K \s_K - J_K \x$ where $W_K$ is a function of $\x$. This means that $\El^K(\x,\s,\la) = \la_K \cdot c_K(\x,\s)$. We then have a quadratic approximation of the entire energy that takes the form

\begin{align*}
E(\x,\s,\la) \approx \tilde{E}(\x,\s,\la) &= E_M(\x^k) + E_\Psi(\s^k) - \El(\x^k,\s^k,\la^k) \\
&+ 
(\x - \x^k)^T \left(
	\frac{\partial E_M}{\partial \x} 
 -  \frac{\partial \C}{\partial \x} \la^k \right) & \\
&+ (\s - \s^k)^T \left(
	\frac{\partial E_\Psi}{\partial \s} 
 -  \frac{\partial \C}{\partial \s} \la^k \right)  & \\
&- (\la - \la^k)^T \C (\x^k, \s^k) & \left(\C (\x^k,\s^k) = \left. \frac{\partial \El}{\partial \la} \right|_{\x^k,\s^k,\la^k}\right) \\
& - (\x - \x^k)^T \left(
  \frac{\partial}{\partial \x} \left(\frac{\partial \C}{\partial \s} \la^k\right) \right)(\s - \s^k) & \\
& - (\x - \x^k)^T \left(
  \frac{\partial \C }{\partial \x} \right)(\la - \la^k) 
- (\s - \s^k)^T \left(
  \frac{\partial \C }{\partial \s} \right)(\la - \la^k) & \\
& + \frac{1}{2} (\s- \s^k)^T \left(\frac{\partial^2 E_\Psi}{\partial \s^2} \right)(\s - \s^k) & \\
& + \frac{1}{2}(\x- \x^k)^T \left(\frac{\partial^2 E_M}{\partial \x^2} -
\frac{\partial}{\partial \x} \left(\frac{\partial \C}{\partial \x} \la^k\right)
\right)(\x - \x^k) & \\
\end{align*}
Next we cancel out some terms and also take the second derivative of $\C$ to be zero and get
\begin{equation}
\begin{split}
\tilde{E}(\x,\s,\la) &=  E_M(\x^k) + E_\Psi(\s^k) + 
(\x - \x^k)^T \left(\frac{\partial E_M}{\partial \x} \right)  \\
&+ (\s - \s^k)^T \left(\frac{\partial E_\Psi}{\partial \s} \right)   
 - \la^T \C (\x^k, \s^k) \\
& - (\x - \x^k)^T \left(
  \frac{\partial}{\partial \x} \left(\frac{\partial \C}{\partial \s} \la^k\right) \right)(\s - \s^k) \\
& - (\x - \x^k)^T \left(
  \frac{\partial \C }{\partial \x} \right)(\la) 
- (\s - \s^k)^T \left(
  \frac{\partial \C }{\partial \s} \right)(\la) \\
& + \frac{1}{2}(\s- \s^k)^T \left(\frac{\partial^2 E_\Psi}{\partial \s^2} \right)(\s - \s^k) \\
& + \frac{1}{2}(\x- \x^k)^T \left(\frac{\partial^2 E_M}{\partial \x^2}
\right)(\x - \x^k), \\
\end{split}
\end{equation}
and now using that $x - x^k = \dx$ and $\s - \s^k = \ds$:
\begin{equation}
\begin{split}
\tilde{E}(\x,\s,\la) &=  E_M(\x^k) + E_\Psi(\s^k) + 
\dx^T \left(\frac{\partial E_M}{\partial \x} \right) 
+ \ds^T \left(\frac{\partial E_\Psi}{\partial \s} \right)   
 - \la^T \C (\x^k, \s^k) \\
& - \dx^T \left(
  \frac{\partial}{\partial \x} \left(\frac{\partial \C}{\partial \s} \la^t\right) \right)\ds
- \left( \dx^T 
  \frac{\partial \C }{\partial \x}
+  \ds^T  
  \frac{\partial \C }{\partial \s} \right) \la \\
&+ \frac{1}{2}\ds^T \left(\frac{\partial^2 E_\Psi}{\partial \s^2} \right)\ds
+ \frac{1}{2} \dx^T \left(\frac{\partial^2 E_M}{\partial \x^2}
\right)\dx. \\
&\\
&=  E_M(\x^k) + E_\Psi(\s^k) + 
\dx^T \gx
+ \ds^T \gs
 - \la^T \C (\x^k, \s^k) \\
& - \dx^T \left(
  \frac{\partial}{\partial \x} \left(\frac{\partial \C}{\partial \s} \la^k\right) \right)\ds
- \left( \dx^T 
  \frac{\partial \C }{\partial \x}
+  \ds^T  
  \frac{\partial \C }{\partial \s} \right) \la \\
& + \frac{1}{2}\ds^T H^k \ds
+ \frac{1}{2h^2} \dx^T M \dx. \\
\end{split}
\end{equation}

\section{System of Equations}
Now to setup our system of equations, we differentiate our energy with respect to the $\delta$-variables and set equal to zero. We switch back to vector notation and make $K$ a subscript to simplify the notation.


\subsection{Equation for $\dx$}
Last we differentiate $\tilde{E}$ with respect to $\dx$ (we omit $\frac{\partial^2 \C_K}{ \partial \x^2} \dx$ for now):

\begin{equation}
\frac{\partial \tilde{E}}{\partial \dx} = \frac{1}{h^2}M\dx + \gx 
- \sum_{K \in \Th} \left(
J_K^T(\hat{W}_K^{k}\cdot \la_K^k) \ds_K +
J_K^T(\hat{W}_K^{k}\cdot \s_K^k - I)\la_K
\right)V_K
\end{equation}


\subsection{Equation for $\ds$}
We start by differentiating $\tilde{E}$ with respect to $\ds$:
\begin{equation}
\frac{\partial \tilde{E}^K}{\partial \ds_K} =  \gs_K + H_K^k \ds_K -
{W_K^k}^T \la_K - (\hat{W}_K^{k}\cdot \la_K^k)^T J_K \dx
\end{equation}
where $(\hat{W}_K^{t}\cdot \la_K^t)$ performs a reduction over the middle index of $\hat{W}^K_{ijk}$. Setting this derivative equal to zero and solving for $\ds_K$ gives
\begin{equation}
\ds_K = {H_K^k}^{-1} \left( {W_K^k}^T \la_K + (\hat{W}_K^{k}\cdot \la_K^k)^T J_K \dx - \gs \right).
\end{equation}

\subsection{Equation for $\dl$}
In the same fashion we differentiate $\tilde{E}$ with respect to $\dl$:

\begin{equation}
\begin{split}
\frac{\partial \tilde{E}^K}{\partial \la_K} &= -\C_K (x^k,\s^k)- \left( \dx^T 
  \frac{\partial \C_K }{\partial \x} +  \ds^T \frac{\partial \C_K }{\partial \s} \right) \\
  &= -\C_K (x^k,\s^k) -(\hat{W}_K^{k}\cdot \s_K^k - I)^T J_K \dx - W_K^k \ds\\
\end{split}
\end{equation}
where $(\hat{W}_K^{k}\cdot \s_K^k)$ performs a reduction over the last index of $\hat{W}^K_{ijk}$.


\subsection{System of Equations}
To build the system of equations we define some variables to keep things neat:
\begin{align}
F &= - \sum_{K \in \Th} {P_K^\s}^T \left((\hat{W}_K^{k}\cdot \la_K^k)^T J_K^w \right) \\
G &= J^w - \sum_{K \in \Th} {P_K^\la}^T \left((\hat{W}_K^{k}\cdot \s_K^k)^T J_K^w \right) \\
D &= -\sum_{K \in \Th} {P_K^\la}^T  W_K^k P_K^\s
\end{align}
where $J^w$ is the volume weighted Jacobian, and $P_K^\la \in \R^{9 \times 9n_e}$ and $P_K^\s \in \R^{6 \times 6n_e}$ are selection matrics for the $K$-th set of Lagrange multipliers and deformation variables, respectively. Additionally $H^k$ is the block diagonal Hessian for $s^k$. We form the system by setting each of the derivatives with respect to the $\delta$-variables equal to zero and get

\begin{equation}
\begin{pmatrix}
\frac{1}{h^2}M & F^T & G^T \\
F & H & D^T\\
G & D & 0
\end{pmatrix}
\begin{pmatrix}
\dx \\
\ds \\
\la
\end{pmatrix} =
\begin{pmatrix}
-\gx \\
-\gs \\
\C (\x^k, \s^k)
\end{pmatrix}.
\end{equation}

Next, to get this thing in a KKT-like form we substitute the solution for the $\ds$ into the other equations. So we have
\begin{equation}
\ds = {H}^{-1}\left(-\gs - D^T \la - F\dx \right),
\end{equation}
and sorry cause it's about to get a little ugly, but for the first set of equations we have
\begin{equation*}
\frac{1}{h^2}M\dx + F^TH^{-1}\left(-\gs - D^T\la - F\dx \right) +G^T \la = -\gx
\end{equation*}
and collect terms to get
\begin{equation}
\left(\frac{1}{h^2}M - F^TH^{-1}F \right)\dx + (G^T - F^TH^{-1}D^T)\la = 
F^TH^{-1}\gs -\gx.
\end{equation}
Next we substitute the solution for $\ds$ into the equations for $\la$:
\begin{equation*}
G\dx + DH^{-1}\left(-\gs - D^T\la - F\dx\right) = \C (\x^k, \s^k)
\end{equation*}
and collect terms so that we have
\begin{equation}
(G - DH^{-1}F)\dx - DH^{-1}D^T\la = \C (\x^k, \s^k) + DH^{-1}\gs.
\end{equation}
Before moving the equations back into matrix form, we define the following variables:
\begin{align}
\tilde{M} &= (\frac{1}{h^2}M - F^TH^{-1}F) \\
\tilde{J} &= (G - DH^{-1}F). 
\end{align}

Now our final system of equations for a single Newton step is
\begin{equation}
\begin{pmatrix}
\tilde{M} & \tilde{J}^T \\
\tilde{J} & -DH^{-1}D^T 
\end{pmatrix}
\begin{pmatrix}
\dx \\
\la
\end{pmatrix} =
\begin{pmatrix}
F^TH^{-1}\gs -\gx \\
\C (\x^k, \s^k) + DH^{-1}\gs
\end{pmatrix}.
\end{equation}

\section{The various numerical methods}

%% This declares a command \Comment
%% The argument will be surrounded by /* ... */
\SetKwComment{Comment}{/* }{ */}

\begin{algorithm}[htp]
\caption{Newtons method}\label{alg:one}
\SetKwFunction{algo}{NewtonStep}
\SetKwProg{myalg}{Algorithm}{}{}
\myalg{\algo{$\x^t,\s^t$, \textup{max\_iters}}}{

\SetKwRepeat{Do}{do}{while}

$\x \gets \x^t$\;
$\s \gets \s^t$\;
$\la \gets 0$\;
$\x^0 \gets \x^t$\;
$\s^0 \gets \s^t$\;
$E^0 \gets \text{energy}(\x,\s,\la)$\;

iter $\gets 0$\;
\Do{not converged \textup{or} \textup{iter} $ < $ \textup{max\_iters}}{

	
	\tcp{Compute system LHS and RHS}
	$H,\mathbf{g} = \text{compute\_gradients}(\x,\s,\la)$\;
	\BlankLine
	\tcp{Descent direction}
	$\dx,\la = -H^{-1} \mathbf{g}$\;
	
	$\alpha \gets 1$\;
	\Do{$\text{energy}(\x,\s,\la)$}{
	$\x \gets \x^0 + \alpha \dx$\;
	$\alpha \gets \alpha / 2$\;
	$\s \gets \s^0 + \text{compute\_ds}(\x,\la) > E^0$\;
    }
	$\x^0 \gets \x$\;
	$\s^0 \gets \s$\;
	$E^0 \gets \text{energy}(\x,\s,\la)$\;
	iter $\gets$ iter $ + 1$\;
}
\Return $\x,\s,\la$

}
\end{algorithm}

\end{document}